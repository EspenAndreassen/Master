\chapter{Introduction}

\pagenumbering{arabic}

\minitoc

The introduction chapter takes a closer look at the motivation behind the study. It also looks at the concrete problem description and the background for this description, as well as the research question. Afterwards, a closer look at the scope and limitations of the research project is performed. Ending the chapter is a section giving a closer look at the report outline.

\newpage

\section{Motivation}

In 2007 an article on the future of socio-technical coordination in global software engineering was published by James D. Herbsleb \cite{Herbsleb2007}. In this work he gathered previous research carried out on the area of coordination and looked at where future studies could focus their attention. It is interesting to notice that already at such an early stage focus on coordination and effective coordination mechanisms in global projects were present. In his research he highlighted that there was a pressing need for deeper understanding of which kind of coordination that will be required in the globalisation witnessed, and which effect this will have on the business world. This article was one of the main motivators for the master thesis.

Organisations and companies around the world are in a transformation phase with a lot of them transitioning from traditional development methodologies to agile approaches. An example is SAP AG moving from a waterfall-like approach to the introduction of Scrum and a lean development style in a large-scale context. From this transition experiences were extracted. A lot of these experiences focused on the complexity of managing multi-team development when scaling Scrum \cite{Nord2011}. This highlights and motivates the need for focus to be directed towards coordination, collaboration and communication studies in such large-scale agile software development.

As will also be seen in this master thesis the large-scale agile development world is not only present globally, but has slowly taken place at a domestic domain as well. This further shows the topical and relevant nature of the study. The master thesis was in that sense made even more interesting for the researcher because of the case company and project taking place in Norway.

\section{Problem Description and Background}
\label{pdab}

Since the introduction of agile development methodologies their usage have seen a steady growth. This has led to an increasing need for studies that reflect on the consequences and different aspects following the paradigm shift. One of these aspects is how coordination is handled \cite{Agerfalk2006, Leffingwell2007, Cockburn2002, Batra2010}. At the International Conference on Agile Software Development (XP2013) ``Inter-team coordination'' was voted the number one burning topic in large-scale agile software development, with ``Large project organization'' coming in second \cite{Dingsoyr2013b}.  In the latest years there has evidentially been an increase in companies and organisations performing development through agile development methodologies in large-scale projects \cite{Paasivaara2012, Com2013, Vlietland2015, Lindvall2004, Dingsoyr2013b, Lee2008, Paasivaara2009}, but the effects have not been well-documented \cite{Pikkarainen2008, Paasivaara2012, Freudenberg2010, Haaster2014, Dingsoyr2013a, Reifer2003}. In the study this topic will be highlighted with the focus on coordination in large-scale agile projects. Theory, literature and models from the Software-field will be used and compared to other fields to see which changes and similarities the paradigm shift has brought forth (theories and literature from large-scale will be used where this is available). Below the research questions are outlined:

\begin{fancyquotes}
Which similarities and dissimilarities in inter-team coordination can be found between current literature on large-scale/MTS projects, and a large-scale agile software development project in practice? And which aspects and mechanisms were identified as important for this inter-team coordination, as well as for general team performance?
\end{fancyquotes}

The purpose of the study and the planned master thesis will therefore be a combination of ``To add to the body of knowledge'', ``To solve a problem'', ``To find the evidence to inform practice'', ''To develop a greater understanding of people and their world'' and ``To contribute to other people's well-being'' \cite{Oates2006}.

While research in small-scale agile software development is starting to get a good track record \cite{Paasivaara2012, Haaster2014}, there is a clear gap in the research surrounding coordination in large-scale agile software development \cite{Pikkarainen2008, Paasivaara2012, Dingsoyr2013b}, and large-scale agile software development in general \cite{Freudenberg2010, Haaster2014}. Therefore, this master thesis will contribute in filling parts of the gap. This will involve ``An exploration of a topic, area or field'', as well as ``An in-depth study of a particular situation'' in the case study \cite{Oates2006}.

As stated above, small-scale agile software development research is starting to get a good track record with successful findings. Because of these findings large organisations have been interested in adopting the benefits agile software development has shown over traditional development methods \cite{Com2013, Vlietland2015, Agerfalk2006, Paasivaara2012}. The assumption that agile methodologies will deliver the same benefits when scaled to larger organisations and projects is therefore an interesting topic.

The combination of filling the gap and looking at the aforementioned assumption will be the pillars in the research outcomes.

\section{Scope, Limitations and Acknowledgement}

As time constraints were put on the master thesis it was obvious that some attention had to be aimed towards the scope of the report and the limitations this would imply. As mentioned in the previous section \ref{pdab} large-scale agile projects, and agile projects in general, are growing in numbers. With this growth a lot of questions and interesting research problems arise. It is therefore important to specify that this particular master thesis only aims to cover the described research questions: ``Which similarities and dissimilarities in inter-team coordination can be found between current literature on large-scale/MTS projects, and a large-scale agile software development project in practice? And which aspects and mechanisms were identified as important for this inter-team coordination, as well as for general team performance?''.

Further, the research project does not aspire to introduce a brand new theory regarding the combination of large-scale, agile software development, coordination and effectiveness. The objective is to find and categorise research performed concerning the combination of these themes and look for common conclusions in their findings, as well as identifying and calling attention to clear gaps that need to be filled in the research field. After this empirical review has been conducted the findings will be compared to a real life case project carried out at a large-scale agile software development project in Norway.

To give some insight and a clearer picture of the study, theory from agile software development, coordination and large-scale will be presented. Findings from a literature review will also be given on the combination of the aforementioned themes. It is important to note that the focus on coordination will primarily be on coordination across teams and not on coordination within these teams.

It is important to acknowledge that some of the work gathered and used in this master thesis was carried out in a preliminary study by the researcher. Some of this work is included in different chapters, e.g., the theory chapter. Also some of this work has been rewritten or made more thorough and elaborated.

Lastly, the research study will not focus on frameworks and electronic tools suggested to support the large-scale agile processes. In this study the focus will rather be aimed towards robust empirical studies performed on the research area of coordination in a large-scale context, as well as the interviews performed with all involved organisations of the Omega-project.

\section{Target Audience}

It is important to have an audience in mind when researching and publishing a work. In this particular master thesis the main effort has been on three types of audiences:

\begin{itemize}
   \item \textbf{Researchers} from the coordination and agile field could find such a research interesting because of the small pool of research articles available on similar work.
   \item \textbf{Practitioners} working with or adopting agile methodologies in a large-scale context might find such a study interesting because it could give insight to possible pitfalls and benefits.
   \item \textbf{Computer science students} could possible find such a work interesting because most of the textbooks and research in general on agile software methodologies and development has only included small-scale projects. It could also give insight to how work is performed in real life projects, and not only how the textbooks describe it.
\end{itemize}

%Researchers, practitioners, fellow computer science students

\section{Report Outline}

\begin{description}
    \item[Chapter 1: Introduction] contains a brief and general introduction to the study at hand and the motivation behind it.
    \item[Chapter 2: Theory] looks at important aspects of the research question, namely software development methodologies, coordination, large-scale, and performance in coordination.
    \item[Chapter 3: Method] explains how the literature review was carried out throughout the research project.
    \item[Chapter 4: Results] outlines the studies selected from the literature review, as well as their findings. It also links these studies to Strode's theoretical model of coordination.
    \item[Chapter 5: Discussion] contains a summarised look at the findings from the results chapter, and connects these to the research question. Strode's theoretical model of coordination is also discussed further with regards to its applicability in a large-scale context.
    \item[Chapter 6: Conclusion] carries out a summary of the most paramount points of the results and discussion chapters.
    \item[Chapter 7: Future Work] outlines possible routes to take in the research field on inter-team coordination.
\end{description}