\chapter{Introduction}

\pagenumbering{arabic}

\minitoc

The introduction chapter takes a closer look at the motivation behind the study. It also looks at the concrete problem description and the background for this description, as well as the research question. Afterwards, a closer look at the scope and limitations of the research project is performed. Ending the chapter is a section giving a closer look at the report outline.

\newpage

\section{Motivation}


\section{Problem Description and Background}
\label{pdab}

\begin{fancyquotes}
...
\end{fancyquotes}

\section{Scope and Limitations}

\section{Report Outline}

\begin{description}
    \item[Chapter 1: Introduction] contains a brief and general introduction to the study at hand and the motivation behind it.
    \item[Chapter 2: Theory] looks at important aspects of the research question, namely software development methodologies, coordination, large-scale, and performance in coordination.
    \item[Chapter 3: Method] explains how the literature review was carried out throughout the research project.
    \item[Chapter 4: Results] outlines the studies selected from the literature review, as well as their findings. It also links these studies to Strode's theoretical model of coordination.
    \item[Chapter 5: Discussion] contains a summarised look at the findings from the results chapter, and connects these to the research question. Strode's theoretical model of coordination is also discussed further with regards to its applicability in a large-scale context.
    \item[Chapter 6: Conclusion] carries out a summary of the most paramount points of the results and discussion chapters.
    \item[Chapter 7: Future Work] outlines possible routes to take in the research field on inter-team coordination.
\end{description}