\chapter{Future Work}


\minitoc

In this chapter possible research for the future is highlighted.
\newpage

\section{Suggestions for Future Research Focus}

With the introduction and growth of using agile approaches in large-scale software development a lot of focus needs to be aimed in this direction. This study has primarily looked at how coordination affects the performance level of such development projects. Through the study some remarks were made.

Firstly, in the practitioner world of agile software development through Scrum the so-called Scrum-of-Scrums are suggested as the coordination mechanism in large-scale projects. However, there has been little evidence of its success in practice. As could be seen from chapter \ref{results} the results from the studies used in this research had several complaints and problems regarding SoS meetings. More effort needs to be focused on this area, for example through testing CoPs as a new mechanism for inter-team coordination.

Further, a look at Strode's theoretical model of coordination's application in a large-scale context needs to be evaluated. As could be seen from earlier chapters some similarities were categorised, but more research has to be committed, especially regarding the synchronisation component, different complexity factors introduced, the coordinator role, and the proximity aspect of the structure component.

In general more solid empirical case studies need to be performed on coordination and its effect on performance achieved in large-scale agile software development, as well as focus on extracting a mechanism that works in practice (which the SoS meetings did not seem to achieve in the studies looked at here).

\newpage