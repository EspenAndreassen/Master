\chapter{Future Work}


\minitoc

In this chapter possible research for the future is highlighted. Several focus areas are described such as the possibility to create a model for coordination mechanisms in a large-scale context. In general more solid empirical case studies have to be performed on the research field.

\newpage

\section{Suggestions for Future Research Focus}

With the introduction and growth of using agile approaches in large-scale software development and multi-team systems a lot of focus needs to be aimed in this direction. This is backed up by results gathered at the International Conference on Agile Software Development (XP2013) where ``inter-team coordination'' was voted the number one burning topic in large-scale agile software development, with ``large project organisation'' coming in second \cite{Dingsoyr2013b}. This study has primarily looked at important coordination mechanisms and other closely related aspect that affects the performance level of such development projects. Through the study some remarks were made.

What was interesting in the research was that several of these coordination mechanisms and aspects seemed to be somewhat correlated. This leads the researcher to believe that there might be possible to build a model for important aspects of inter-team coordination in large-scale development and multi-team systems. However, for this to be realised more empirical case studies have to be carried out on similar projects to see if the different mechanisms and aspects are reoccurring.

Hence, it would have been interesting to perform a similar study on a different large-scale agile development project and comparing the findings and results to this particular study. For future research a multi-case study could be a possibility to gain more solid data.

As aforementioned it would be interesting to see if it is possible to build a concrete model for mechanism and aspect involved in inter-team coordination and general team performance. A possibility could be to use Strode's theoretical model of coordination as a platform, but focusing on a large-scale context (seeing as this model is built from case studies on small-scale agile software development projects). Especially the the synchronisation component, different complexity factors introduced by large-scale, the coordinator role, and the proximity aspect of the structure component must be looked into closely.

In earlier chapters it was identified a dissimilarity between previous research on informal communication and what was witnessed in the Omega-project and brought forward by the case interviews. This is an interesting remark, and might be an area for further research too analyse if it was a one time ting occurring in this project, or if this is something repeating itself in other successful large-scale projects.

As one of the project leaders at Alpha also pointed out, it is harder to recreate the less concrete coordination mechanisms and aspects. This might also be an area which could benefit from future research focus, in the sense that making these important aspects easier to replicate will serve practitioners in adopting these in their development, leading to better performance standards.

In general more solid empirical case studies need to be performed on coordination mechanisms and related aspect, and their effect on performance achieved in multi-team system and large-scale agile software development.

\newpage