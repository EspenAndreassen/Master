\renewcommand{\abstractname}{Sammendrag}
\begin{abstract}

I senere år har smidig utviklingmetodikk sett en stadig vekst. Smidige tilnærminger var opprinnelig utviklet for små-skala kontekster for å dekke det økende behovet for fleksibilitet og trangen etter å være først ute på markedet med teknologier i konstant forandring. Fordelene identifisert i små-skala smidig utvikling har åpnet øynene til større organisasjoner. Derfor har  det ikke vært overraskende å se at stor-skala smidige programvareutviklingsprosjekter velger å gå for smidig utviklingmetodikk. Likevel viser det seg at forskning innenfor smidige utviklingmetodikker innenfor en stor-skala kontekst er mangelfull.

Et annet aspekt som har sett et økende fokus i de senere årene har vært koordinering og koordineringeffektivitet, som er identifisert som viktige faktorer i programvareutvikling og oppnåelse av teamprestasjoner.

Disse to aspektene er kombinert og videre forsket på i dette forskningstudiet. Fokuset vil være på solide empiriske studier tidligere utført på koordinering i stor-skala smidig programvareutviklingsprosjekter, i tillegg til en utforskende case-studie ved et norsk stor-skala smidig programvareutviklingsprosjekt. Funnene fra denne case-studien vil bli sammenlignet med funn fra tidligere publisert arbeid og teori på koordinering.

Hovedfunnene viste hovedsakelig likheter mellom eksisterende studier og arbeidet utført i denne masteroppgaven. De mest fremtredende koordineringmekanismene og aspektene identifisert til å ha en positiv effekt på koordinering og teamprestasjonnivå var samlokalisering, uformell kommunikasjon,  tilstedeværelse fra prosjektledelse og prosjekteier, kontinuerlig endring og forbedring, gjensidig tillit, og felles mentale modeller. Det var derimot noen ulikheter i forhold til eksisterende studier tilstede. Den mest synlige av disse var hvorfor uformelle kommunikasjonarenaer var i stor grad eksisterende i case-prosjektet, selv om noe tidligere litteratur og forskning hadde konkludert med at slike arenaer skulle minke i bruk når team- og prosjektstørrelse ble stor.

\textbf{Nøkkelord:} Stor-skala; MTS; Multi-team-systemer; Koordinering; Koordineringeffektivitet; Smidig; Programvareutvikling; Systemutvikling; Scrum; Gjensidig Tillit; Felles Mentale Modeller; Samlokalisering; Uformell Kommunikasjon; Kontinuerlig Endring og Forbedring; Tilstedeværelse fra Ledelse

\end{abstract}