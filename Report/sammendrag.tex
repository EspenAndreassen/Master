\begin{abstract}

In later years agile development methodologies have seen a steady growth. Agile approaches were originally developed for small-scale contexts to cover the increasing need for flexibility and the urge to be first-to-market with technology in constant change. The benefits witnessed in this small-scale adoption has got large organisations to open their eyes. Therefore, it has not been surprising to see large-scale software development projects opt for the use of agile methodologies. However, the research regarding agile development in a large-scale context is still scarce.

Another aspect that has seen an increasing focus in the later years has been coordination effectiveness, which is identified as an important factor in software development and team performance.

These two aspects are combined and looked further into in this research study. The focus is on robust empirical studies performed on coordination in large-scale agile software development projects. Strode's theoretical model of coordination is also looked further into to identify its applicability in a large-scale context.

The main findings show similarities to coordination effectiveness in small-scale agile software development, but also some dissimilarities. Synchronisation, team co-location, an organisational openness culture, and appropriate infrastructure and supportive tools seem to have a positive effect on the team performance. On the other hand, number of sites and team size, as well as large time zone differences between teams, seem to have a negative effect on the level of effectiveness achieved through coordination in large-scale agile software development projects.

\textbf{Keywords:} Large-scale; Coordination; Coordination Effectiveness; Agile Software Development; Scrum

\end{abstract}