\chapter{Method}

\minitoc

The research study used different methods to gather the relevant publications that were selected. These are further outlined in this chapter starting with a detailed look at the literature review performed, as well as highlighting other parts of the gathering methodology, namely snowball sampling and general accumulation of papers.

\newpage

\section{Literature Review}

%Databases: Web of science (ISI), ACM, Science Direct (Elsevier), Google Scholar
%Keywords:
%Coordination, communication, collaboration
%Large-scale, multiteam, global
%Efficiency, Effectiveness, Productivity, Performance
%Location (co-located, collocated, colocated), distributed, dispersed
%Agile

For this study a literature review was chosen as the information gathering method. For the searching process and selection of articles in the literature review certain recommendations from systematic reviews were followed. The general procedure of such a review is outlined in L1 below. It is important to note that the searching had an open-mindedness regarding search words and the selection process.

\vspace{0.5cm}

\fbox{\parbox{\textwidth}{L1 - The steps of a systematic review \cite{khan2003}:
\begin{enumerate}
  \item Framing questions for a review.
  \item Identifying relevant work.
  \item Assessing the quality of studies.
  \item Summarizing the evidence.
  \item Interpreting the findings.
\end{enumerate}}}

\vspace{0.5cm}

Some of the benefits and objectives of a literature review are summarised in L2 below.

\vspace{0.5cm}

\fbox{\parbox{\textwidth}{L2 - Objectives of a literature review \cite{Oates2006}:
\begin{itemize}
  \item Show that the researcher is aware of existing work in the chosen topic area.
  \item Place the researcher's work in the context of what has already been published.
  \item Point to strengths, weaknesses, omissions or bias in the previous work.
  \item Identify key issues or crucial questions that are troubling the research community.
  \item Point to gaps that have not previously been identified or addressed by researchers.
  \item Identify theories that the researcher will test or explore by gathering data from the field.
  \item Suggest theories that might explain data the researcher has gathered from the field.
  \item Identify theories, genres, methods or algorithms that will be incorporated in the development of a computer application.
  \item Identify research methods or strategies that the researcher will use in the research.
  \item Enable subsequent researchers to understand the field and the researcher's work within that field.
\end{itemize}}}

\vspace{0.5cm}

\subsection{General Outline}

As explained in subsection \ref{general} a set of articles and publications were provided by the supervisor to give an overview on the field and agile software development in general. This made it easier to classify which studies to look for and how to evaluate their relevance and rigour. The databases used in the literature review are summarised in table \ref{databases}. When searching in these databases concepts and keywords were combined to match the research question as well as other interesting combinations. These concepts and keywords are outlined in table \ref{searchwords}. It is important to note that the last concept was an additional search word used because a lot of research seemed to either have focused on a co-located or a distributed manner.

\begin{table}[H]
\begin{center}
    \begin{tabular}{ | p{5cm} | p{8cm} |}
    \hline
    \textbf{Name} & \textbf{Impact} \\ \hline
    ISI Web of Science & apps.webofknowledge.com \\ \hline
    ACM Digital Library & dl.acm.org  \\ \hline
    Science Direct (Elsevier) & sciencedirect.com \\ \hline
    Google Scholar & scholar.google.com \\ \hline
    \end{tabular}
    \caption{Databases used in the literature review.}
    \label{databases}
\end{center}
\end{table}

\begin{table}[H]
\begin{center}
    \begin{tabular}{ | p{5cm} | p{8cm} |}
    \hline
    \textbf{Concept} & \textbf{Keywords} \\ \hline
    Coordination & Communication, collaboration \\ \hline
    Agile & Scrum, XP, Crystal, Lean, Extreme Programming, Xtreme Programming  \\ \hline
    Large-scale & Global, multiteam/multi-team, distributed, international \\ \hline
    Effectiveness & Efficiency, productivity, performance \\ \hline
    Location (Additional search words) & Co-located, collocated, colocated, co located, distributed, dispersed, global, globally  \\ \hline
    \end{tabular}
    \caption{Search words used in the literature review.}
    \label{searchwords}
\end{center}
\end{table}

The literature review provided an extensive amount of findings, unfortunately a lot of the publications were focusing on small-scale development. Therefore, a selection process had to be carried out. Here all abstracts of the collected literature were read and publications with the highest relevance were chosen. The articles that were still left after this selection process were then read thoroughly where some were discarded to give an appropriate amount of publications. The analysis outlined above focused mainly on finding articles focusing on large-scale agile inter-team coordination, meaning such articles were given a higher score when identified. Some other aspects that contributed to the score were mentioning of global projects, effectiveness and inter-team coordination in general. This process was important because of the time constraints specified on the study, and to obtain relevant and rigorous literature to insure a robust study.

\subsection{Snowball Sampling}

Snowball sampling is a term that reflects how new studies are selected through already chosen studies (based on their similarities) \cite{Goodman1961}. This was done in two ways in the research. In table \ref{databases} a list of databases used for the literature review are summarised. Some of these databases provided snowball sampling in the way of suggesting similar articles when a specific publication was selected from a search. This is the first way of snowballing used. The second way was through using reference lists in selected articles and publications. This extraction lead to a lot of well-written and recognised papers.

\subsection{General Accumulation}
\label{general}

Articles were also accumulated through a supervisor and fellow research students. At the start a handful of publications were received from the supervisor, and other papers were also acquired throughout the study. It is important to note that all the articles were inspected in the same manner as the publications found from the literature review to make sure their relevance and rigour were appropriate.