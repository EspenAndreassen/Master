\chapter{Method}

\minitoc

The research study used different methods to gather the relevant publications that were selected. These are further outlined in this chapter starting with a detailed look at the literature review performed, as well as highlighting other parts of the gathering methodology, namely snowball sampling and general accumulation of papers.

\newpage

\section{Literature Review}

%Databases: Web of science (ISI), ACM, Science Direct (Elsevier), Google Scholar
%Keywords:
%Coordination, communication, collaboration
%Large-scale, multiteam, global
%Efficiency, Effectiveness, Productivity, Performance
%Location (co-located, collocated, colocated), distributed, dispersed
%Agile

For this study a literature review was chosen as the information gathering method. For the searching process and selection of articles in the literature review certain recommendations from systematic reviews were followed. The general procedure of such a review is outlined in L1 below. It is important to note that the searching had an open-mindedness regarding search words and the selection process.

\vspace{0.5cm}

\fbox{\parbox{\textwidth}{L1 - The steps of a systematic review \cite{khan2003}:
\begin{enumerate}
  \item Framing questions for a review.
  \item Identifying relevant work.
  \item Assessing the quality of studies.
  \item Summarizing the evidence.
  \item Interpreting the findings.
\end{enumerate}}}

\vspace{0.5cm}

Some of the benefits and objectives of a literature review are summarised in L2 below.

\vspace{0.5cm}

\fbox{\parbox{\textwidth}{L2 - Objectives of a literature review \cite{Oates2006}:
\begin{itemize}
  \item Show that the researcher is aware of existing work in the chosen topic area.
  \item Place the researcher's work in the context of what has already been published.
  \item Point to strengths, weaknesses, omissions or bias in the previous work.
  \item Identify key issues or crucial questions that are troubling the research community.
  \item Point to gaps that have not previously been identified or addressed by researchers.
  \item Identify theories that the researcher will test or explore by gathering data from the field.
  \item Suggest theories that might explain data the researcher has gathered from the field.
  \item Identify theories, genres, methods or algorithms that will be incorporated in the development of a computer application.
  \item Identify research methods or strategies that the researcher will use in the research.
  \item Enable subsequent researchers to understand the field and the researcher's work within that field.
\end{itemize}}}

\vspace{0.5cm}

\subsection{General Outline}

As explained in subsection \ref{general} a set of articles and publications were provided by the supervisor to give an overview on the field and agile software development in general. This made it easier to classify which studies to look for and how to evaluate their relevance and rigour. The databases used in the literature review are summarised in table \ref{databases}. When searching in these databases concepts and keywords were combined to match the research question as well as other interesting combinations. These concepts and keywords are outlined in table \ref{searchwords}. It is important to note that the last concept was an additional search word used because a lot of research seemed to either have focused on a co-located or a distributed manner.

\begin{table}[H]
\begin{center}
    \begin{tabular}{ | p{5cm} | p{8cm} |}
    \hline
    \textbf{Name} & \textbf{Impact} \\ \hline
    ISI Web of Science & apps.webofknowledge.com \\ \hline
    ACM Digital Library & dl.acm.org  \\ \hline
    Science Direct (Elsevier) & sciencedirect.com \\ \hline
    Google Scholar & scholar.google.com \\ \hline
    \end{tabular}
    \caption{Databases used in the literature review.}
    \label{databases}
\end{center}
\end{table}

\begin{table}[H]
\begin{center}
    \begin{tabular}{ | p{5cm} | p{8cm} |}
    \hline
    \textbf{Concept} & \textbf{Keywords} \\ \hline
    Coordination & Communication, collaboration \\ \hline
    Agile & Scrum, XP, Crystal, Lean, Kanban, Extreme Programming, Xtreme Programming  \\ \hline
    Large-scale & Global, multiteam/multi-team (systems), distributed, international \\ \hline
    Effectiveness & Efficiency, productivity, performance \\ \hline
    Location (Additional search words) & Co-located, collocated, colocated, co located, distributed, dispersed, global, globally, international  \\ \hline
    \end{tabular}
    \caption{Search words used in the literature review.}
    \label{searchwords}
\end{center}
\end{table}

The literature review provided an extensive amount of findings, unfortunately a lot of the publications were focusing on small-scale development. Therefore, a selection process had to be carried out. Here all abstracts of the collected literature were read and publications with the highest relevance were chosen. The articles that were still left after this selection process were then read thoroughly where some were discarded to give an appropriate amount of publications. The analysis outlined above focused mainly on finding articles focusing on large-scale agile inter-team coordination, meaning such articles were given a higher score when identified. Some other aspects that contributed to the score were mentioning of global projects, effectiveness and inter-team coordination in general. This process was important because of the time constraints specified on the study, and to obtain relevant and rigorous literature to insure a robust study.

\subsection{Snowball Sampling}

Snowball sampling is a term that reflects how new studies are selected through already chosen studies (based on their similarities) \cite{Goodman1961}. This was done in two ways in the research. In table \ref{databases} a list of databases used for the literature review are summarised. Some of these databases provided snowball sampling in the way of suggesting similar articles when a specific publication was selected from a search. This is the first way of snowballing used. The second way was through using reference lists in selected articles and publications. This extraction lead to a lot of well-written and recognised papers.

\subsection{General Accumulation}
\label{general}

Articles were also accumulated through a supervisor and fellow research students. At the start a handful of publications were received from the supervisor, and other papers were also acquired throughout the study. It is important to note that all the articles were inspected in the same manner as the publications found from the literature review to make sure their relevance and rigour were appropriate.

\section{Research Method}



\fbox{\parbox{\textwidth}{L3 -  TYPE OF CASE STUDY \cite{}:

}}

\subsection{Case Selection}
\label{case_selection}

Before the case study was conducted several criteria for a fitting case project were agreed upon. Seeing the research was suppose to focus on large-scale development/multiteam systems it was important to find a case where minimum two teams were present in the project, as well as collaborating across the teams. It was also important that the project performed in the case was an agile software development project. Another criteria was that the length of the project had to be suitable, meaning that the project had been ongoing for quite some time. The reasoning behind this was that the amount of data would be larger, and it would be easier to find patterns over a longer period of time.

There were also other criteria which were preferable, but not mandatory. One of these criteria was that it would be desirable if there were several roles within the project as a whole and the project teams. This had to do with the possibility of people with different roles within a project having various experiences from the course of the project leading to valuable data, or put in other words, having different points of view within the project. Another preferred criteria was having a large-scale project with several organisations involved. With several organisations involved there will be different cultures and protocols involved, and therefore a lot of interesting data could surface when comparing the approach of the different organisations.

\subsection{Data Collection}

For the data collection in the exploratory case study focus groups were selected. In these focus groups aspects that are known to be challenging in large-scale agile software development were brought up, as well as general discussion on the topic of large-scale software development. Focus groups are further outlined in L4, and were primarily selected because of their ability to accumulate extensive and valuable amounts of research data.

\fbox{\parbox{\textwidth}{L4 -  FOCUS GROUP \cite{}:

}}

In total three focus groups were conducted, one for each of the organisations involved. The topic that was looked at in the focus groups was ``Inter-team coordination and knowledge sharing''. The reasoning behind conducting focus groups for each of the organisations, and not performing them on a project level, was to make sure that an openness was achieved, and that data concerning specific organisations were not lost. As mentioned in section \ref{case_selection} there might be differences in cultures and methodologies between organisations, and these might not have been present in the focus groups if they were held on a joint basis.

The organisations were asked to provide their most relevant personnel to attend each focus group. In total 8 participants were involved in the focus groups. The participants had several roles in the Omega-project: development managers, scrum masters, (sub-)project managers, developers, delivery managers, functional architects and technical architects. It is important to note that most of the focus group participants were employees in management positions in the project. Most of these participants started as developers before switching to management roles with the course of the project. Because of the availability of personnel and topic in the focus groups no pure developers were present. The distribution of participants in the different focus groups is summarised in table \ref{pifg}.

\begin{table}[H]
\begin{center}
    \begin{tabular}{ | p{5.5cm} | p{4cm} | p{5cm} |}
    \hline
    \textbf{Theme} & \textbf{Organisation} & \textbf{Number of participants} \\ \hline
    \multirow{3}{*}{} & Alpha & 2 \\ \cline{2-3}
    Inter-team coordination and knowledge sharing & Beta & 3 \\ \cline{2-3}
    & Gamma & 3 \\ \hline
    \end{tabular}
    \caption{Participants in focus groups.}
    \label{pifg}
\end{center}
\end{table}

%TODO: Legg ved apendix til interview guides

Before the focus group sessions were conducted interview guides were developed, as well as a rough timeline of the project. The timeline was used to freshen the participants' memories about key events. In the focus groups the role of the researcher was to moderate the discussion and take notes. At the start of each focus group the participants were asked to explain their role(s) in the Omega-project. All of the focus group meetings were recorded digitally and transcribed at a later point in time, and whiteboard drawings were documented through pictures. In total the three focus groups resulted in 94 pages of transcribed material. Minutes of each focus group was also sent to each of the corresponding participants for needed information and review.

\subsection{Data Analysis}

