\chapter{Conclusion}
\label{concl}

\minitoc

In this chapter the findings from both the results- and discussion-chapter are further looked at and summarised. The focus of the conclusion-chapter is to make verdicts and decisions based on both previous research and the performed case study to answer the initial research questions.

\newpage

\section{Research Questions}

\begin{fancyquotes}
Which similarities and dissimilarities in inter-team coordination can be found between current literature on large-scale/MTS projects, and a large-scale agile software development project in practice? And which aspects and mechanisms were identified as important for this inter-team coordination, as well as for general team performance from the studied case project?
\end{fancyquotes}

As can be seen from table \ref{cmuatwo}, \ref{cmuasito} and \ref{ocmaia} several coordination mechanisms and other influential aspects were identified in the Omega-project. Some of these were further investigated and more thoroughly analysed in chapter \ref{disc}. Looking back at chapter \ref{theory}, \ref{results} and \ref{disc} involving theory, results and discussion some conclusions regarding the research questions can be drawn.

In general the large-scale agile development project that has been studied showed a lot of similarities to previously conducted research and their findings. However, some dissimilarities were present. The most prominent of these was Cockburn's statement that informal communication should decrease with the project and team size becoming large leading to a decline in communication effectiveness \cite{Cockburn2000}. This was not the case in Omega where the informal communication arenas actually seemed to increase throughout the course of the project leading to improved communication, coordination and collaboration, despite of the project size being so large

To summarise the findings of the master thesis and to answer the second part of the research question a brief description of each identified coordination mechanism or other important aspects are outlined in table \ref{conclusion_final}. It is important to note that all of these factors and sources are deemed to have a positive impact on coordination effectiveness and general team performance, meaning a lack of their presence will have a negative effect.

%\begin{table}[H]
\begin{center}
	\begin{longtable}{| p{3.4cm} | p{1.25cm} | p{6.4cm} |}

    \hline \textbf{Mechanism/Aspect} & \textbf{Impact} & \textbf{Description of impact on inter-team coordination and performance} \\ \hline
    \endfirsthead

    \multicolumn{3}{c}%
{{\bfseries \tablename\ \thetable{} -- continued from previous page}} \\ \hline
   \textbf{Mechanism/Aspect} & \textbf{Impact} & \textbf{Description of impact on inter-team coordination and performance} \\ \hline
    \endhead

    \multicolumn{3}{|r|}{{Continued on the next page\ldots}} \\ \hline
    \endfoot

   \endlastfoot 

	Co-location & \textcolor{ForestGreen}{Positive} & Co-location led to improved communication quality and generally more communication being achieved, especially on an informal level. This in turn improved the general performance levels of both teams and the project as a whole. Being co-located was seemingly the most dominant positive factor witnessed in the project, and closely linked to several of the other mechanisms and aspects identified, e.g., mutual trust and informal communication arenas. \\ \hline
	Informal communication & \textcolor{ForestGreen}{Positive} & The most dominant communication type was informal communication. The use of such arenas grew throughout the project leading to better flow in inter-team coordination, collaboration and communication. Because of this, better knowledge, general overview and status of the project was achieved elevating the performance standard within Omega. \\ \hline
	Leadership presence & \textcolor{ForestGreen}{Positive} & Also identified as an important aspect of the Omega-project was the presence of both the project management and the project owner. With such leadership present so closely to the development teams inter-team coordination and general problem solving could be handled on a fast paced level excelling team and project performance. This aspect seemed to be closely linked to both co-location and mutual trust. \\ \hline
	Mutual trust & \textcolor{ForestGreen}{Positive} & Another important mechanism brought up in the case interviews was mutual trust. Mutual trust was mainly achieved through social gatherings and co-location such that members got to know each other leading to a unity. This links mutual trust to both co-location and informal communication, but also seemed to be closely related to shared mental models and the presence of project management (as this increased trust between the project management and developers). In general the achieved mutual trust led to team members more easily communicating with each other improving coordination and collaboration, as well as general performance in both the teams and the project as a whole. \\ \hline
	Shared mental models & \textcolor{ForestGreen}{Positive} & An aspect identified as having a positive impact on inter-team coordination and performance levels of the project was shared mental models. As the project moved along team members seemed to get a better understanding of other members leading to their mental models being more equal. It was especially knowledge and experience sharing arenas that led to achieving such shared mental models. With the team members getting a better understanding of others, inter-team coordination levels improved, as well as general performance standards. \\ \hline
	Continuous change and improvement & \textcolor{ForestGreen}{Positive} & In Omega there was a focus on adapting based on a need basis to achieve more efficient work. The impact of this on inter-team coordination was that communication arenas were in constant change to improve how coordination and general collaboration was handled. \\ \hline
	\caption{Summary of identified impacts on coordination and performance.}
	\label{conclusion_final}
	\end{longtable}
\end{center}
%\end{table}