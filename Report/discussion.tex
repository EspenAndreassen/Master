\chapter{Discussion}
\label{disc}

\minitoc

In this chapter a closer look...
\newpage

\section{Research Question}

The results in chapter \ref{results} outlined in detail several aspects identified from the three interviews carried out at the case project. These results will be further discussed in this chapter, focusing on comparing the results with appropriate theory and literature mainly described in chapter \ref{theory}. The discussion will revolve around the research question for the master thesis:

\begin{fancyquotes}
Which similarities and dissimilarities in inter-team coordination can be found between current literature on large-scale/MTS projects, and a large-scale agile software development project in practice?
\end{fancyquotes}

\subsection{Co-location}

It was evident from the interviews that co-location played a big part in achieving a high level of efficiency, both in a development productivity aspect, as well as a coordination aspect. Several interviews brought up the factor of co-location and highlighted that some teams located only a building away faced lack of communication. As one of the project leaders at Beta put it:

\begin{fancyquotes}
I think being co-located was a big advantage, especially having all the teams located on the same floor and space. If you are, e.g., located at each side of a town or building it would be a barrier for communication.
\end{fancyquotes}

This is definitely in line with previous findings in similar research. As identified in studies on the field outlined in section \ref{large_scale_coordination} several of these pointed out that co-location had a positive impact on coordination efficiency and team performance in general. Both Melo et al. \cite{Melo2013} and Dingsøyr et al. \ref{Dingsoyr2013c} highlighted the correlation between co-location and coordination effectiveness.

\subsection{Informal Communication Arenas}

Another aspect that was brought up by several of the interviewees was the extensive use of informal communication within the Omega-project. One of the Scrum masters at Gamma highlighted an interesting thought that the wide-ranging use of informal communication arenas might have been present because of the teams being co-located. This gives the researcher belief that there could be a connection between the two aspects. This view is shared by Cockburn in an article on project methodology selection \cite{}. He states that:

%http://www.eee.metu.edu.tr/~bilgen/Cockburn647.pdf

\begin{fancyquotes}
The most effective form of communication (for transmitting ideas) is interactive and face-to-face, as at a whiteboard.
\end{fancyquotes}

His work implies that co-located developers will have more frequent communication and therefore have a higher productivity level than people being dispersed. It is interesting that he brought up whiteboards as an example, as this was something several interviews highlighted as important for communication (especially informal communication). Dingsøyr et al. \ref{Dingsoyr2013c} also pointed at the importance of such visualising tools which can be seen in table \ref{closedloop}. However, Cockburn goes on to say that as project and team size increases the informal communication should decrease, ending in communication effectiveness going down. In the case project at hand this does not seem to be the case. Even though there are several teams and personnel involved the informal communication was present to a large degree, and at times seemed to be more efficient than formal communication arenas.

Some of the interviews pointed out that there seemed to be less need for the formal meeting-places because members got to know each other better throughout the project. It is important to note that with the increase in use of informal communication arenas witnessed in the Omega-project, this did not mean formal communication was not important or present. As one of the project leaders at Beta highlighted:

\begin{fancyquotes}
I think you need both [informal and formal arenas], but without the informal communication and the common determination to work things out, then I don't think large-scale projects will work. However, I don't think you can manage to control such a project well enough with only formal channels.
\end{fancyquotes}

It seems as if a good balance between formal and informal communication arenas are a key part in achieving a high level of coordination efficiency and team performance. However, it seems like informal communication will have a larger weight on the scale in this balance. This is in line with Mintzberg's \cite{} coordination mechanism called ``mutual adjustment'' which states that ``members coordinate their own work by informal communication with each other'', and seems to be the coordination mechanism most present in agile development.

\subsection{Continuous Change and Improvement}

What also seemed to be an important factor in the project was how well the teams, team members and project as a whole adapted based on needs. In a way all levels of the project could be seen as self-organising, in line with the agile mindset. In particular communication and coordination, as well as knowledge sharing, arenas seemed to experience continuous change. A project leader at Gamma put it this way:

\begin{fancyquotes}
We adjusted which meetings were used on a need basis within the project. Some arenas were present throughout the whole project, while other came and went. I believe this was important. [...] E.g., we could identify in a Metascrum-meeting that there were areas which needed more, or less, coordination.
\end{fancyquotes}

Having such a mindset focusing on adaptation might not be as easy in all projects, especially for companies which are not used to working with agile methodologies. Therefore this is an area where more focus could be directed because the adaptation policies seemed to achieve higher project performance on several levels. The matter at hand was nicely summed up by one of the interviewees:

\begin{fancyquotes}
New meeting arenas were spawned with time passing, but others were removed when the need ceased to exist. It is the presence of an agile mindset to change based on needs.
\end{fancyquotes}

It is important to note that there is a somewhat linkage between the changing of coordination arenas and the increase in use of informal communication, as well as trust and shared mental models which will be discussed in section \ref{mtasmm}. This has to do with the project members finding it easier to keep a fast paced communication flow and the communication getting better as the members got to know each other, leading to the discovery of communication arenas that were both needed and redundant.

\subsection{Presence from Project Management and Owner}

Moving on to the fourth identified aspect was how the presence of both the project management and project owner affected the project. Having the project management co-located with the developers seemed to have a big impact on general performance within the project. Because the different members of the project management teams were present it was easier for them to maintain a good overview of the status and relevant information throughout the course of the project. Therefore it gives the researcher reason to believe there is a correlation between co-location of the project management with the development teams, and the performance level achieved in Omega. The researcher also believes there are ties between the presence of project management, and the increasing use of informal communication channels. With the the project management being co-located they could coordinated and communicate on a more regular and freely basis with other project personnel. One of the project leaders at Gamma illustrated it this way:

\begin{fancyquotes}
Within the teams, at least something I tried, was minimum having one conversation with each of the Scrum master every day. Doing this I knew what everyone was doing so I could prioritise correctly with the information I gathered. Having this information you could act as an ``information carrier'' which made decision making and problem solving easier.
\end{fancyquotes}

Further some of the interviewees pointed out the importance of having the customer involved closely with the project. As a couple of project leaders at Alpha pointed out the project owner had representatives typically available 95\% of the time. And these representatives had authority to make decisions. This is backed up by some of Strode's \cite{Strode2012} propositions of coordination effectiveness showed in section \ref{strode_ce}. Proposition 1a states the following:

\begin{fancyquotes}
\textbf{Proposition 1a:} A coordination strategy that includes synchronisation and structure coordination mechanisms improves project coordination effectiveness when the customer is included in the project team. Synchronisation activities and associated artefacts are required at all frequencies – project, iteration, daily, and ad hoc.
\end{fancyquotes}

Further proposition 3, which is closely linked to co-location as well, states that close proximity, high availability, and high substitutability will increase implicit coordination effectiveness. This was definitely the case in the Omega-project where especially close proximity and high availability were focused on, but some measures were also taken to achieve higher substitutability, such as Beta trying to convert specialised teams towards more general teams so they could collaborated and substitute if needed.

It was also noted by one of the project leaders at Alpha that at times when there were higher complexity levels in the project the bosses from the project owner were present. This is also something included in Strode's work covered by proposition 5 which states that to maintain coordination effectiveness when facing high project complexity the frequency of iteration and ad hoc synchronisation activities should be increased. Below a quote from the mentioned project leader is included:

\begin{fancyquotes}
When we faced tougher stages throughout the project several bosses from the project owner were present. They walked around and talked to people as well, so it was not only us doing that.
\end{fancyquotes}

One of the architects at Beta also expressed his belief that the present from the project management team helped build trust between them and the rest of the project members. This shows possibilities for a connection between the presence of the project management, and mutual trust, which will be further discussed in the coming section.

\subsection{Mutual Trust and Shared Mental Models}
\label{mtasmm}

The last of the five identified aspects from the case interviews was mutual trust and shared mental models. These two aspects seem to closely linked to each other. Previous research on both areas have shown that they have an extensive impact on communication, collaboration and coordination, as well as team and project performance in general \cite{bandow, salas, cooper, mathieu}. Mathieu et al. \cite{}, e.g., found that there was a notably positively correlation between task-work mental model similarity and teamwork mental model similarity, and team process, which in turn were to a large degree connected to team performance.

As stated in chapter \ref{results} there were problems with an individualistic focus from project teams at an early stage of the project. This led to teams not aiming their attention towards the total delivery of the project. After this was handled, meaning project management changed the mindset of the teams towards a shared understanding, that productivity and collaboration increased. Which is supported by the previously mentioned research. As one of the project leaders at Alpha put it:

\begin{fancyquotes}
It is all about optimising the totality. A team is better than an individual member, and a sub-project is better than each of the teams on their own.
\end{fancyquotes}

%TODO CITE?
In Omega the experience and knowledge exchanging arenas played a large part in adopting shared mental models. Especially ``pair programming'' seemed to be a key factor. Some of the benefit with this practice seem to be improved production, better code quality, enhanced job-satisfaction, increased knowledge sharing, and team building and improved communication. Some, if not all, were evident at the Omega-project.

Another factor that seemed to affect trust-building was social arenas outside of the office. Team members often went out together, e.g., to celebrate the end of a sprint. These social gatherings seemed to have a positive effect on how members perceived others. Dingsøyr et al. \cite{Dingsoyr2013c} found similar findings in their focus group study where they deemed ``social atmosphere'' as an important sub-component of closed-loop communication which in turn was important for team performance. This is summarised in table \ref{closedloop}. One of the interviewees expressed his thoughts on the matter of social gatherings and their impact on the project:

\begin{fancyquotes}
There were several social gatherings, e.g., trips out on town. I believe that a lot of things happened there which were not directly visible, but in turn were important for project members gaining trust and building a better unity.
\end{fancyquotes}

Another factor that seemed to have an impact on mutual trust and shared mental models was the openness culture witnessed in the project. Again this was something identified by Dingsøyr et al. \cite{Dingsoyr2013c} as being important to achieve a higher level of performance, and is summarised in table \ref{closedloop}. This leads the researcher to believe that there might be a connection between informal communication and mutual trust.

Lastly, co-location also was identified as a key aspect for gaining better understanding and trust between project members. Which again highlights the importance of co-location in such projects. Two project leaders at Alpha shared their views on the matter explaining how they thought being located in the same office space made it easier to become a unity. It lead to members knowing each other, gaining a better shared mental model regarding aspects of the project.

\subsection{Summary}

%Sammenlign med tabell 2.6? Co-location, Openness culture, Infrastructure (bad office facilities?), Social atmosphere, Visualising status and progress (whiteboards!)
%Bruk den om at det er vanskelig å gjenskape slike mindre håndfaste ting

As can be seen from the previously described section several important aspects for increasing coordination and general performance were identified. As can also be witnessed most of these aspects have previously been identified as important both in small-scale and large-scale projects, as well as in agile and software development, and other fields and industries. Looking at the research question several aspects have in turn been shown to be similar in a large-scale/MTS agile software development project, and in previous research on inter-team coordination and coordination in general.

As shown in table \ref{closedloop} and \ref{summary} showing the summary of impacts on team performance and coordination from previously conducted studies, several of these aspects are identified in this case study, e.g., co-location having a positive impact. However, some dissimilarities were found. The most prominent of these were regarding informal communication. Cockburn \cite{} stated that with a large team and project size the informal communication levels should be lower, ending in communication effectiveness decreasing. This was not the case in this large-scale project where the informal communication arenas seemed to increase throughout the course of the project leading to improved communication, coordination and collaboration.

What could be important to note for practitioners is that some of the aspects identified in this study in general are harder to replicate than more concrete aspects. This was highlighted by one of the project leaders at Alpha. He stated the following:

\begin{fancyquotes}
It is easy to recreate Metascrums, Scrum of Scrums and the experience and knowledge sharing. The concrete, specific aspects are easy to replicate, but the team dynamics, having everyone located at the same floor, constant communication, the togetherness witnessed, and similar things, the less concrete aspects, they are harder to reproduce.
\end{fancyquotes}

\section{Evaluation of the Study}

\subsection{Research Process}

%\subsection{}